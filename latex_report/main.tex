\documentclass{mldsmsc}

\title{Deep Reinforcement Learning for Ad Personalization}
\author{Martin Bat\v{e}k}
\CID{00951537}
\supervisor{Mikko Pakkanen}
\date{1 May 2023}
%For today's date, use:
%\date{\today}
\logoimg{}


% THIS IS WHERE NEW COMMANDS CAN BE DEFINED
% commands below only used in the proof; otherwise can be deleted
\newcommand{\consta}{a}
\newcommand{\X}{X}
\newcommand{\EE}[1]{ \mathrm{E} [ #1 ] }
\newcommand{\inparenth}[1]{\left( #1 \right)}

\begin{document}

% Generates the Title Page
\maketitle


% Generates plagiarism declaration
\declarationname{Martin Bat\v{e}k}
\declarationdate{17 July 2024}
\declaration 


\begin{abstract}
    ABSTRACT GOES HERE
\end{abstract}

\begin{acknowledgements}
    ANY ACKNOWLEDGEMENTS GO HERE
\end{acknowledgements}

% add glossary?

% table of contents
\tableofcontents

% VERY IMPORTANT
% This command switches from Roman to Arabic numbering for main part of thesis
\mainmatter


\chapter{Introduction}

The global digital advertising market is worth approximately \$602 billion today. Due to the increasing rate of of online participation since the 
COVID-19 pandemic, this number has been rapidly increasing and is expected to reach \$871 billion by the end of 2027 \citep{RefWorks:emarketer2023digital}.
Many of the of the major Ad platforms such as Google, Facebook and Amazon operate on a cost-per-user-engagement pricing model, which usually means that 
advertisers get charged for every time a user clicks on an advertisment. This means that these platforms are incentivized to make sure that the content 
shown to each user is as relevent as possible in order to maximize the number of clicks in the long term. Attaining accurate Click-Through Rate (CTR) 
prediction is a necessary first step for Ad persionalization, which is why study of CTR prediction methods have been an extremely active part of 
Machine Learning research over the past through years.

Initially, shallow prediction methods such as Logistic Regression, Factorization Machines \citep{RefWorks:rendle2010factorization} and Field-Aware Factorization 
Machines \citep{RefWorks:juan2016field-aware} have been used for CTR prediction. However, these methods have often been shown to be unable to capture the 
higher order feature interactions in the sparse multi-value categorical Ad Marketplace datasets \citep{RefWorks:zhang2021deep}. Since then, Deep Learning methods have been 
shown to show superior predictive ability on these datasets. The focus of my reasearch project is therefore to explore the merits of different Deep 
Learning architechtures for click-through rate prediction.

Paragraph about Reinforcement Learning and how it can be used for Ad Personalization.

Paragraph about the structure of the thesis.

\chapter{Background}

Background chapter.

\section{Problem Statement}

Section content goes here. 

\section{Literatiure Review}

\subsection{Deep CTR Prediction}

\subsection{Deep Reinforcement Learning}

\chapter{Deep CTR model Evaluation}

\section{Model Selection Methodology}

\section{Model Summaries}

\subsection{Shallow Models}

\subsubsection{Logistic Regression}

\subsubsection{Factorization Machines}

\subsection{Deep Models}

\subsubsection{Factirization Supported Neural Networks}

\subsubsection{Product Based Neural Networks}

\subsubsection{Wide \& Deep Learning}

\subsubsection{DeepFM}

\subsubsection{Feature Generation by Convolutional Neural Network}

\subsubsection{Automatic Feature Interaction Learning}

\section{Benchmark Datasets and Exploratory Data Analysis}

\section{Model Evaluation}

\section{Deep CTR Model Results}

\chapter{Deep Reinforcement Learning for Ad Personalization}

\section{DeepCTR-RL Framework}

\section{Experiment Setup}

\section{Results}

\chapter{Discussion}

Discussion goes here.

\chapter{Conclusion}


Conclusion goes here. 





\clearpage
 %% reset page counter and start appendix pages with A
\pagenumbering{arabic}
\renewcommand*{\thepage}{A\arabic{page}}

%% Appendix goes here
%\appendix
%
%\chapter{Appendix title}
%
%Appendix goes here.


%%References part of appendices
% References: modify the file refs.bib
\bibliographystyle{plainnat}
\bibliography{refs}


\end{document}
